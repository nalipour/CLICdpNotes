\section{Simulation framework}\label{sec:Framework}

The software chain used to obtain the results shown in this note is the following: the detector geometry is described in a compact XML format which is interpreted by GeomConverter~2.4~\cite{GeomConverter-homepage}. The SLIC v3r0p3 program~\cite{Graf:2006ei} used to simulate the detector is a thin wrapper around the \textsc{Geant4}~\cite{Agostinelli2003250} package. The digitisation and several steps of the event reconstruction are performed using the org.lcsim 2.5 framework~\cite{lcsim-homepage}. Calorimeter clustering and particle flow analysis are performed by PandoraPFA~\cite{Thomson:2009rp, Marshall:2012ry}. Finally, LCFIPlus v00-05-02~\cite{website:LCFIPlus} is used for flavour tagging. The reconstruction of primary and secondary vertices and the jet finding based on the Durham algorithm~\cite{Catani:1991hj} are also handled by this package. \textsc{IlcDirac}~\cite{ilcdirac} is used to perform the analysis on the computing grid. \\
In the study presented in this note, beam-induced backgrounds are not included. This choice is made to keep the performance of the vertex-detector geometry separate from the impact of the beam-induced backgrounds. In addition, the required computing time of about 150 CPU years remains at a manageable level. 

%The software chain used for the results shown in this note is defined as follows: first, the detector geometry is described using GeomConverter version 2.4 \cite{GeomConverter-homepage}. Then the SLIC package \begin{it}v3r0p3\end{it} \cite{slic-homepage} (a \textsc{Geant4} \cite{Agostinelli2003250} based simulation software) is used to simulate the interaction of the particles with the detector. Finally LCSim version 2.5 \cite{lcsim-homepage}  handles the reconstruction and the analysis of the tracks of the particles. The PandoraPFA \cite{Thomson:1192027} is used as a particle flow analysis combining tracking and calorimetry.  
%After the reconstruction of particles tracks, the LCFIPlus version \begin{it}v00-05-02\end{it} \cite{website:LCFIPlus} package is used to perform vertex and jet finding and also flavour tagging. Each step is implemented as a Marlin processor \cite{marlin} (version \begin{it}v01-04\end{it}).
