\section{Conclusions}\label{sec:Conclusions}
A new detector model for CLIC is under development which takes into account the progressing engineering studies while respecting the physics requirements. \\
Two new layouts for the CLIC vertex detector have been implemented in simulation models: the \textit{spirals} and the \textit{double\_spirals} geometries. The spiral arrangement of the modules in the vertex endcaps allows to use airflow cooling which has the potential to reduce the material budget significantly. The \textit{double\_spirals} geometry, with the same amount of material budget as the CDR detector, provides more sensitive layers. These two layouts show similar impact parameter resolutions as the CDR geometry.\\
Flavour tagging is investigated for simulated dijet events using the newly implemented and the CDR geometries. The b-tagging or c-tagging efficiencies versus the fake rates are compared. The overall results show that the implemented geometries are similar in terms of the flavour-tagging performance. We observe that for dijet events with a polar angle $\theta$ of around $40^\circ$, the flavour tagging degrades for the spiral geometries compared to the CDR geometry. For this polar angle, the number of sensors for the spiral geometries depends on the azimuthal angle $\phi$. This effect is worse for double-layered sensors and might be largely compensated using optimisations for the pattern recognition depending on the $\phi$-angle. \\
From these results we conclude that the \textit{spirals} and \textit{double\_spirals} geometries, a priori, are equally suitable for the vertex detector at CLIC. \\
The effect of the material budget on the flavour-tagging performance has also been studied. The fake rates increase by approximately 5-35\% when increasing the amount of material per double layer from $0.2\%X_{0}$ to $0.4\%X_{0}$. \\
The effect of the flavour-tagging performance on a physics analysis is illustrated for the process e$^+$e$^-\rightarrow$ H$\nu\bar{\nu}$ at 3~TeV with subsequent decays H$\rightarrow$b\={b} and H$\rightarrow$c\={c}. Changes of the fake rates by $\pm$20\% lead to changes in the precisions of the $\sigma \times$BR measurements by $\pm$6-7\% for H$\rightarrow$b\={b} and by $\pm$15\% for H$\rightarrow$c\={c}. \\
In future extensions of this study, beam-induced backgrounds could be included. This would increase the required computing resources considerably. 

