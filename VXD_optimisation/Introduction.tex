\section{Introduction}\label{sec:Introduction}

The precision physics and experimental conditions at CLIC set challenging requirements on the vertex detector: excellent spatial resolution ($\sim$\SI{3}{\micro\meter} single-point resolution per layer), time slicing of hits with $\sim$\SI{10}{\nano\second} precision, geometrical coverage extending down to low polar angles ($\theta_{min}\approx~8^{\circ}$), extremely low mass ($\sim~0.2\%X_{0}$ per detection layer, including readout, support and cabling) and efficient heat removal from sensors and readout. These considerations push the technology beyond the limits of current vertex detectors. \\
In the CLIC Conceptual Design Report (CDR)~\cite{CLICCDR2012}, simplified vertex detector geometries were used which do not fully take into account the above-mentioned requirements. For example, airflow cooling is considered as a strategy to significantly reduce the amount of material. However, the CDR geometries do not provide a path for the air to flow through the detector. Moreover, the total amount of material per detection layer is optimistically assumed to be $0.1\%X_{0}$ for the CDR geometries, as detailed models for the cabling and support were not available. \\
In this study, different options for the vertex barrel and endcap layouts are considered. A spiral arrangement of the sensors in the vertex endcap regions is implemented, allowing for air flow through the vertex-detector volume~\cite{CoolingSimulations}. Geometries with double-layer arrangements are compared to single-layer layouts. Finally, a geometry with increased material budget is implemented based on engineering studies for supports and cabling. \\
The study is based on full detector simulations using \textsc{Geant4}. The performance of each geometry is evaluated by investigating the beauty and charm tagging efficiencies for different jet energies and polar angles. The performances of the implemented geometries are compared~\cite{AlipourTehrani:1606436}.
% using the LCFIPlus package \cite{website:LCFIPlus}
